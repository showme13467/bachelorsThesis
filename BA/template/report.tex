\documentclass[12pt,a4paper,twoside]{report}
\input{Talk1/MyHeader}

\usepackage[german,english]{babel}
\usepackage[T1]{fontenc} 
\usepackage[latin1]{inputenc}
\usepackage{amsfonts}
\usepackage{amsmath}
\usepackage{latexsym}
\usepackage{amssymb}
\usepackage{capt-of}
\usepackage{epsfig}
\usepackage{moreverb}
\usepackage{rotating}
\usepackage{enumerate}
\usepackage{graphics, graphicx,wrapfig}
\usepackage{fancybox}
\usepackage{picinpar,varioref,floatflt}
\usepackage{ae}
\usepackage{longtable}
\usepackage{textcomp}
\usepackage{unizhtr}
\usepackage{chapterbib}
\usepackage{minitoc}
\usepackage{float}
\usepackage{url}
\usepackage{caption}
\usepackage{longtable}
\captionsetup{format=hang,margin=10pt,font=footnotesize,labelfont=bf,justification=raggedright}
\captionsetup[table]{position=top}
\setlength{\abovecaptionskip}{5pt}
\setlength{\belowcaptionskip}{0pt}
\usepackage{listings}
\lstset{
    basicstyle=\ttfamily\tiny, %\scriptsize
    %frame=single, % adds a frame around the code
    xleftmargin=4em,
    xrightmargin=3em,
}




%%%%%%%%%%%%%%%%%%%%%%%%%%%%%%%%%%%%%%%%%%%%%%%%%%

% Disable widow and orphan lines
\clubpenalty=10000
\widowpenalty=10000
\raggedbottom

\setcounter{secnumdepth}{3} 

\selectlanguage{english}
\setcounter{tocdepth}{0}
\pagestyle{myheadings}

%%%%%%%%%%%%%%%%%%%%%%%%%%%%%%%%%%%%%%%%%%%%%%%%%%
% Margin settings if the cover page is not used
% (Comment them out if \maketitle and \makeimprint are used)
% \textheight240mm
% \setlength{\oddsidemargin}{4mm}
% \setlength{\evensidemargin}{-5.5mm}
% \topmargin-6.0mm
% \setlength{\parindent}{0ex}
% \setlength{\parskip}{2.0ex plus 0.9ex minus 0.4ex}
%%%%%%%%%%%%%%%%%%%%%%%%%%%%%%%%%%%%%%%%%%%%%%%%%%

\begin{document}
\dominitoc

%%%%%%%%%%%%%%%%%%%%%%%%%%%%%%%%%%%%%%%%%%%%%%%%%%

% Define the authors printed on the cover page
\author{Alexander Nu�baum}
% Define the authors printed in the imprint
\authorshort{Alexander Nu�baum}
% Define the authors' group within IFI
\authorgroup{ }%Institut f\"ur Technische Informatik (INF)}
% Define the authors webpage
\authorwebpage{}
% Define the title with optional subtitle
\title{Bachelorarbeit}
% Define the title on the back of the title page
\titleimprint{Bachelorarbeit}
% Define the publishing date of the report, 1=January, 2=February, ...
\reportmonth=1
\reportyear{2020}
% Define the technical report number
\reportnumber{INF3 - 2020/1}
% Define the language of the cover page, 0=german, 1=english
\reportenglish{1}

%%%%%%%%%%%%%%%%%%%%%%%%%%%%%%%%%%%%%%%%%%%%%%%%%%

% Make the title page	
\maketitle

% Make the imprint on the back of the cover page
\makeimprint

\setlength{\parindent}{0ex}

% Include the file(s) for the technical report



%\selectlanguage{german}
\tableofcontents
\cleardoublepage

%-------------------------------
%Example Report 
%-------------------------------
{
%\selectlanguage{german}
%\renewcommand{\mtctitle}{Inhaltsverzeichnis}
\selectlanguage{english}
\renewcommand{\mtctitle}{Contents}
}


\cleardoublepage

%------------------------------------------------------------------------------------------------------------------
%YOUR Report
%------------------------------------------------------------------------------------------------------------------
%In case you like to write in German start without any changes in the following
%       -  For the German special characters use the following format: 
%          Ä =  {\"A}, O = {\"O}, Ü = {\"U}, a = {\"a}, ö = {\"o},  ü = {\"u}, and ß = {\ss}
%In case you like to write in English 
%       - comment out lines 178 and 179
%       - and delete the comment sign "%" in lines 180 and 181
%
%To compile your report compile the file report.tex in the general folder!
%
%Some further regulations for writing the report:
%	  - The abstract should be minimum 1/4 page long.
%       - The Introduction section NEEDS to end with a subsection giving information 
%          about what the reader will expect to read in the rest of the report.
%.      - No subfolders for figures are allowed. Put the direct in the folder Talk1
%.      - To get code examples for Latex feel free to consult the Talk0 source code 
%         for that or google. 
%       - Use label ("\label") in order to make references dynamic. This should be 
%         done for sections, figures, tables, and listings.
%       - In case you refer to a section followed by the corresponding number 
%         directly behind you write it in the following manner: Section 1.1. 
%         Same holds for figures, tables and listings
%.      - Each figure, table and listing must be cross-referred in text and the main 
%         message needs to be mentioned.
%.      - The caption of figure, table and listing is NOT allowed to be longer than 1 line.
%       - For algorithms use the package listing and NOT another one.
%       - References HAVE to be done in a .bib file called literaturtalk1.bib that is already 
%         in your talk folder.
%       - DO NOT use
%              - any additional packages at all.
%			 - For Algorithms use the package "listing".
%              	- For mathematic expressions use the package "amsmath", supporting 
%.                       also the classic math-mode in Latex.
%              - any appendix in your report at all.
%       	- floating text around figures at all. 
%              - the construct "minipages".
%.      	- " \\" at the end of paragraphs. a spaceline does the same. 
%                Just use " \\" if you like to have no spaceline, meaning the content of two 
%                paragraphs belongs context-wise together. 
%
%Remarks/Experiences: 
%.      - In case you use online tools for compiling Latex trouble can occur. This is 
%         known for Overleaf. 
%.      - In case you do not have installed Latex tools on your system you can use 
%         citrix and login in to the Servers of RZ. The machine powered there has 
%         Latex installed and also has a link to you host system's memory.
%-------------------------------


{
\selectlanguage{german}
\renewcommand{\mtctitle}{Inhaltsverzeichnis}
%\selectlanguage{english}
%\renewcommand{\mtctitle}{Contents}
\chapter{title}
\markboth{title}{}
\chaptauthors{Name}

\newcommand{\cmark}{\ding{51}}%
\newcommand{\xmark}{\ding{55}}%

\Kurzfassung{
Fingerprinting ...
}

\newpage

\minitoc %table of contents

\newpage



%From Here indifidual Report content ending with a Section named CONCLUSION

\newpage

\section{Introduction}
\label{intro}
Fingerprinting ...

\subsection{Structure of the Chapter} %in case you have no subsections at all in the Intro-chapter delete this and the next line and write the paragraph of the document structure in normal paragraph style.
\label{structure}
% German example
%Dieses Kapitel ist daher wie folgt strukturiert: Kapitel X.X beschreibt ... In Kapitel X.Y werden Hintergrundinformationen zu den Themen ... gegeben. usw. Abschließend fasst Kapitel \ref{conclusion} dieses Kapitel zusammen und zeigt die gewonnenen Erkenntnisse auf.

%English version. Please make the cross-references dynamic
This chapter is therefore structured as follows: Chapter X.X describes ... Chapter X.Y contains background information on the topics ... etc. etc. Finally chapter \ref{conclusion} summarizes this chapter and shows the insights gained.

\section{XYZ}
\label{xyz}
This ...

\subsection{ABC}
\label{abc}
hello ...


\section{Conclusion}
\label{conclusion}
Fingerprinting ...


%NOT CHANGES from here onwards please!
%References
\bibliographystyle{IEEEtran}
\bibliography{literaturtalk1}

}

\end{document}
