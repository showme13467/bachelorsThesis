\chapter{title}
\markboth{title}{}
\chaptauthors{Name}

\newcommand{\cmark}{\ding{51}}%
\newcommand{\xmark}{\ding{55}}%

\Kurzfassung{
Fingerprinting ...
}

\newpage

\minitoc %table of contents

\newpage



%From Here indifidual Report content ending with a Section named CONCLUSION

\newpage

\section{Introduction}
\label{intro}
Fingerprinting ...

\subsection{Structure of the Chapter} %in case you have no subsections at all in the Intro-chapter delete this and the next line and write the paragraph of the document structure in normal paragraph style.
\label{structure}
% German example
%Dieses Kapitel ist daher wie folgt strukturiert: Kapitel X.X beschreibt ... In Kapitel X.Y werden Hintergrundinformationen zu den Themen ... gegeben. usw. Abschließend fasst Kapitel \ref{conclusion} dieses Kapitel zusammen und zeigt die gewonnenen Erkenntnisse auf.

%English version. Please make the cross-references dynamic
This chapter is therefore structured as follows: Chapter X.X describes ... Chapter X.Y contains background information on the topics ... etc. etc. Finally chapter \ref{conclusion} summarizes this chapter and shows the insights gained.

\section{XYZ}
\label{xyz}
This ...

\subsection{ABC}
\label{abc}
hello ...


\section{Conclusion}
\label{conclusion}
Fingerprinting ...


%NOT CHANGES from here onwards please!
%References
\bibliographystyle{IEEEtran}
\bibliography{literaturtalk1}
